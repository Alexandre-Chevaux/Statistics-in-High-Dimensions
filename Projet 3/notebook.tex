
% Default to the notebook output style

    


% Inherit from the specified cell style.




    
\documentclass[11pt]{article}

    
    
    \usepackage[T1]{fontenc}
    % Nicer default font (+ math font) than Computer Modern for most use cases
    \usepackage{mathpazo}

    % Basic figure setup, for now with no caption control since it's done
    % automatically by Pandoc (which extracts ![](path) syntax from Markdown).
    \usepackage{graphicx}
    % We will generate all images so they have a width \maxwidth. This means
    % that they will get their normal width if they fit onto the page, but
    % are scaled down if they would overflow the margins.
    \makeatletter
    \def\maxwidth{\ifdim\Gin@nat@width>\linewidth\linewidth
    \else\Gin@nat@width\fi}
    \makeatother
    \let\Oldincludegraphics\includegraphics
    % Set max figure width to be 80% of text width, for now hardcoded.
    \renewcommand{\includegraphics}[1]{\Oldincludegraphics[width=.8\maxwidth]{#1}}
    % Ensure that by default, figures have no caption (until we provide a
    % proper Figure object with a Caption API and a way to capture that
    % in the conversion process - todo).
    \usepackage{caption}
    \DeclareCaptionLabelFormat{nolabel}{}
    \captionsetup{labelformat=nolabel}

    \usepackage{adjustbox} % Used to constrain images to a maximum size 
    \usepackage{xcolor} % Allow colors to be defined
    \usepackage{enumerate} % Needed for markdown enumerations to work
    \usepackage{geometry} % Used to adjust the document margins
    \usepackage{amsmath} % Equations
    \usepackage{amssymb} % Equations
    \usepackage{textcomp} % defines textquotesingle
    % Hack from http://tex.stackexchange.com/a/47451/13684:
    \AtBeginDocument{%
        \def\PYZsq{\textquotesingle}% Upright quotes in Pygmentized code
    }
    \usepackage{upquote} % Upright quotes for verbatim code
    \usepackage{eurosym} % defines \euro
    \usepackage[mathletters]{ucs} % Extended unicode (utf-8) support
    \usepackage[utf8x]{inputenc} % Allow utf-8 characters in the tex document
    \usepackage{fancyvrb} % verbatim replacement that allows latex
    \usepackage{grffile} % extends the file name processing of package graphics 
                         % to support a larger range 
    % The hyperref package gives us a pdf with properly built
    % internal navigation ('pdf bookmarks' for the table of contents,
    % internal cross-reference links, web links for URLs, etc.)
    \usepackage{hyperref}
    \usepackage{longtable} % longtable support required by pandoc >1.10
    \usepackage{booktabs}  % table support for pandoc > 1.12.2
    \usepackage[inline]{enumitem} % IRkernel/repr support (it uses the enumerate* environment)
    \usepackage[normalem]{ulem} % ulem is needed to support strikethroughs (\sout)
                                % normalem makes italics be italics, not underlines
    

    
    
    % Colors for the hyperref package
    \definecolor{urlcolor}{rgb}{0,.145,.698}
    \definecolor{linkcolor}{rgb}{.71,0.21,0.01}
    \definecolor{citecolor}{rgb}{.12,.54,.11}

    % ANSI colors
    \definecolor{ansi-black}{HTML}{3E424D}
    \definecolor{ansi-black-intense}{HTML}{282C36}
    \definecolor{ansi-red}{HTML}{E75C58}
    \definecolor{ansi-red-intense}{HTML}{B22B31}
    \definecolor{ansi-green}{HTML}{00A250}
    \definecolor{ansi-green-intense}{HTML}{007427}
    \definecolor{ansi-yellow}{HTML}{DDB62B}
    \definecolor{ansi-yellow-intense}{HTML}{B27D12}
    \definecolor{ansi-blue}{HTML}{208FFB}
    \definecolor{ansi-blue-intense}{HTML}{0065CA}
    \definecolor{ansi-magenta}{HTML}{D160C4}
    \definecolor{ansi-magenta-intense}{HTML}{A03196}
    \definecolor{ansi-cyan}{HTML}{60C6C8}
    \definecolor{ansi-cyan-intense}{HTML}{258F8F}
    \definecolor{ansi-white}{HTML}{C5C1B4}
    \definecolor{ansi-white-intense}{HTML}{A1A6B2}

    % commands and environments needed by pandoc snippets
    % extracted from the output of `pandoc -s`
    \providecommand{\tightlist}{%
      \setlength{\itemsep}{0pt}\setlength{\parskip}{0pt}}
    \DefineVerbatimEnvironment{Highlighting}{Verbatim}{commandchars=\\\{\}}
    % Add ',fontsize=\small' for more characters per line
    \newenvironment{Shaded}{}{}
    \newcommand{\KeywordTok}[1]{\textcolor[rgb]{0.00,0.44,0.13}{\textbf{{#1}}}}
    \newcommand{\DataTypeTok}[1]{\textcolor[rgb]{0.56,0.13,0.00}{{#1}}}
    \newcommand{\DecValTok}[1]{\textcolor[rgb]{0.25,0.63,0.44}{{#1}}}
    \newcommand{\BaseNTok}[1]{\textcolor[rgb]{0.25,0.63,0.44}{{#1}}}
    \newcommand{\FloatTok}[1]{\textcolor[rgb]{0.25,0.63,0.44}{{#1}}}
    \newcommand{\CharTok}[1]{\textcolor[rgb]{0.25,0.44,0.63}{{#1}}}
    \newcommand{\StringTok}[1]{\textcolor[rgb]{0.25,0.44,0.63}{{#1}}}
    \newcommand{\CommentTok}[1]{\textcolor[rgb]{0.38,0.63,0.69}{\textit{{#1}}}}
    \newcommand{\OtherTok}[1]{\textcolor[rgb]{0.00,0.44,0.13}{{#1}}}
    \newcommand{\AlertTok}[1]{\textcolor[rgb]{1.00,0.00,0.00}{\textbf{{#1}}}}
    \newcommand{\FunctionTok}[1]{\textcolor[rgb]{0.02,0.16,0.49}{{#1}}}
    \newcommand{\RegionMarkerTok}[1]{{#1}}
    \newcommand{\ErrorTok}[1]{\textcolor[rgb]{1.00,0.00,0.00}{\textbf{{#1}}}}
    \newcommand{\NormalTok}[1]{{#1}}
    
    % Additional commands for more recent versions of Pandoc
    \newcommand{\ConstantTok}[1]{\textcolor[rgb]{0.53,0.00,0.00}{{#1}}}
    \newcommand{\SpecialCharTok}[1]{\textcolor[rgb]{0.25,0.44,0.63}{{#1}}}
    \newcommand{\VerbatimStringTok}[1]{\textcolor[rgb]{0.25,0.44,0.63}{{#1}}}
    \newcommand{\SpecialStringTok}[1]{\textcolor[rgb]{0.73,0.40,0.53}{{#1}}}
    \newcommand{\ImportTok}[1]{{#1}}
    \newcommand{\DocumentationTok}[1]{\textcolor[rgb]{0.73,0.13,0.13}{\textit{{#1}}}}
    \newcommand{\AnnotationTok}[1]{\textcolor[rgb]{0.38,0.63,0.69}{\textbf{\textit{{#1}}}}}
    \newcommand{\CommentVarTok}[1]{\textcolor[rgb]{0.38,0.63,0.69}{\textbf{\textit{{#1}}}}}
    \newcommand{\VariableTok}[1]{\textcolor[rgb]{0.10,0.09,0.49}{{#1}}}
    \newcommand{\ControlFlowTok}[1]{\textcolor[rgb]{0.00,0.44,0.13}{\textbf{{#1}}}}
    \newcommand{\OperatorTok}[1]{\textcolor[rgb]{0.40,0.40,0.40}{{#1}}}
    \newcommand{\BuiltInTok}[1]{{#1}}
    \newcommand{\ExtensionTok}[1]{{#1}}
    \newcommand{\PreprocessorTok}[1]{\textcolor[rgb]{0.74,0.48,0.00}{{#1}}}
    \newcommand{\AttributeTok}[1]{\textcolor[rgb]{0.49,0.56,0.16}{{#1}}}
    \newcommand{\InformationTok}[1]{\textcolor[rgb]{0.38,0.63,0.69}{\textbf{\textit{{#1}}}}}
    \newcommand{\WarningTok}[1]{\textcolor[rgb]{0.38,0.63,0.69}{\textbf{\textit{{#1}}}}}
    
    
    % Define a nice break command that doesn't care if a line doesn't already
    % exist.
    \def\br{\hspace*{\fill} \\* }
    % Math Jax compatability definitions
    \def\gt{>}
    \def\lt{<}
    % Document parameters
    \title{Projet3}
    
    
    

    % Pygments definitions
    
\makeatletter
\def\PY@reset{\let\PY@it=\relax \let\PY@bf=\relax%
    \let\PY@ul=\relax \let\PY@tc=\relax%
    \let\PY@bc=\relax \let\PY@ff=\relax}
\def\PY@tok#1{\csname PY@tok@#1\endcsname}
\def\PY@toks#1+{\ifx\relax#1\empty\else%
    \PY@tok{#1}\expandafter\PY@toks\fi}
\def\PY@do#1{\PY@bc{\PY@tc{\PY@ul{%
    \PY@it{\PY@bf{\PY@ff{#1}}}}}}}
\def\PY#1#2{\PY@reset\PY@toks#1+\relax+\PY@do{#2}}

\expandafter\def\csname PY@tok@w\endcsname{\def\PY@tc##1{\textcolor[rgb]{0.73,0.73,0.73}{##1}}}
\expandafter\def\csname PY@tok@c\endcsname{\let\PY@it=\textit\def\PY@tc##1{\textcolor[rgb]{0.25,0.50,0.50}{##1}}}
\expandafter\def\csname PY@tok@cp\endcsname{\def\PY@tc##1{\textcolor[rgb]{0.74,0.48,0.00}{##1}}}
\expandafter\def\csname PY@tok@k\endcsname{\let\PY@bf=\textbf\def\PY@tc##1{\textcolor[rgb]{0.00,0.50,0.00}{##1}}}
\expandafter\def\csname PY@tok@kp\endcsname{\def\PY@tc##1{\textcolor[rgb]{0.00,0.50,0.00}{##1}}}
\expandafter\def\csname PY@tok@kt\endcsname{\def\PY@tc##1{\textcolor[rgb]{0.69,0.00,0.25}{##1}}}
\expandafter\def\csname PY@tok@o\endcsname{\def\PY@tc##1{\textcolor[rgb]{0.40,0.40,0.40}{##1}}}
\expandafter\def\csname PY@tok@ow\endcsname{\let\PY@bf=\textbf\def\PY@tc##1{\textcolor[rgb]{0.67,0.13,1.00}{##1}}}
\expandafter\def\csname PY@tok@nb\endcsname{\def\PY@tc##1{\textcolor[rgb]{0.00,0.50,0.00}{##1}}}
\expandafter\def\csname PY@tok@nf\endcsname{\def\PY@tc##1{\textcolor[rgb]{0.00,0.00,1.00}{##1}}}
\expandafter\def\csname PY@tok@nc\endcsname{\let\PY@bf=\textbf\def\PY@tc##1{\textcolor[rgb]{0.00,0.00,1.00}{##1}}}
\expandafter\def\csname PY@tok@nn\endcsname{\let\PY@bf=\textbf\def\PY@tc##1{\textcolor[rgb]{0.00,0.00,1.00}{##1}}}
\expandafter\def\csname PY@tok@ne\endcsname{\let\PY@bf=\textbf\def\PY@tc##1{\textcolor[rgb]{0.82,0.25,0.23}{##1}}}
\expandafter\def\csname PY@tok@nv\endcsname{\def\PY@tc##1{\textcolor[rgb]{0.10,0.09,0.49}{##1}}}
\expandafter\def\csname PY@tok@no\endcsname{\def\PY@tc##1{\textcolor[rgb]{0.53,0.00,0.00}{##1}}}
\expandafter\def\csname PY@tok@nl\endcsname{\def\PY@tc##1{\textcolor[rgb]{0.63,0.63,0.00}{##1}}}
\expandafter\def\csname PY@tok@ni\endcsname{\let\PY@bf=\textbf\def\PY@tc##1{\textcolor[rgb]{0.60,0.60,0.60}{##1}}}
\expandafter\def\csname PY@tok@na\endcsname{\def\PY@tc##1{\textcolor[rgb]{0.49,0.56,0.16}{##1}}}
\expandafter\def\csname PY@tok@nt\endcsname{\let\PY@bf=\textbf\def\PY@tc##1{\textcolor[rgb]{0.00,0.50,0.00}{##1}}}
\expandafter\def\csname PY@tok@nd\endcsname{\def\PY@tc##1{\textcolor[rgb]{0.67,0.13,1.00}{##1}}}
\expandafter\def\csname PY@tok@s\endcsname{\def\PY@tc##1{\textcolor[rgb]{0.73,0.13,0.13}{##1}}}
\expandafter\def\csname PY@tok@sd\endcsname{\let\PY@it=\textit\def\PY@tc##1{\textcolor[rgb]{0.73,0.13,0.13}{##1}}}
\expandafter\def\csname PY@tok@si\endcsname{\let\PY@bf=\textbf\def\PY@tc##1{\textcolor[rgb]{0.73,0.40,0.53}{##1}}}
\expandafter\def\csname PY@tok@se\endcsname{\let\PY@bf=\textbf\def\PY@tc##1{\textcolor[rgb]{0.73,0.40,0.13}{##1}}}
\expandafter\def\csname PY@tok@sr\endcsname{\def\PY@tc##1{\textcolor[rgb]{0.73,0.40,0.53}{##1}}}
\expandafter\def\csname PY@tok@ss\endcsname{\def\PY@tc##1{\textcolor[rgb]{0.10,0.09,0.49}{##1}}}
\expandafter\def\csname PY@tok@sx\endcsname{\def\PY@tc##1{\textcolor[rgb]{0.00,0.50,0.00}{##1}}}
\expandafter\def\csname PY@tok@m\endcsname{\def\PY@tc##1{\textcolor[rgb]{0.40,0.40,0.40}{##1}}}
\expandafter\def\csname PY@tok@gh\endcsname{\let\PY@bf=\textbf\def\PY@tc##1{\textcolor[rgb]{0.00,0.00,0.50}{##1}}}
\expandafter\def\csname PY@tok@gu\endcsname{\let\PY@bf=\textbf\def\PY@tc##1{\textcolor[rgb]{0.50,0.00,0.50}{##1}}}
\expandafter\def\csname PY@tok@gd\endcsname{\def\PY@tc##1{\textcolor[rgb]{0.63,0.00,0.00}{##1}}}
\expandafter\def\csname PY@tok@gi\endcsname{\def\PY@tc##1{\textcolor[rgb]{0.00,0.63,0.00}{##1}}}
\expandafter\def\csname PY@tok@gr\endcsname{\def\PY@tc##1{\textcolor[rgb]{1.00,0.00,0.00}{##1}}}
\expandafter\def\csname PY@tok@ge\endcsname{\let\PY@it=\textit}
\expandafter\def\csname PY@tok@gs\endcsname{\let\PY@bf=\textbf}
\expandafter\def\csname PY@tok@gp\endcsname{\let\PY@bf=\textbf\def\PY@tc##1{\textcolor[rgb]{0.00,0.00,0.50}{##1}}}
\expandafter\def\csname PY@tok@go\endcsname{\def\PY@tc##1{\textcolor[rgb]{0.53,0.53,0.53}{##1}}}
\expandafter\def\csname PY@tok@gt\endcsname{\def\PY@tc##1{\textcolor[rgb]{0.00,0.27,0.87}{##1}}}
\expandafter\def\csname PY@tok@err\endcsname{\def\PY@bc##1{\setlength{\fboxsep}{0pt}\fcolorbox[rgb]{1.00,0.00,0.00}{1,1,1}{\strut ##1}}}
\expandafter\def\csname PY@tok@kc\endcsname{\let\PY@bf=\textbf\def\PY@tc##1{\textcolor[rgb]{0.00,0.50,0.00}{##1}}}
\expandafter\def\csname PY@tok@kd\endcsname{\let\PY@bf=\textbf\def\PY@tc##1{\textcolor[rgb]{0.00,0.50,0.00}{##1}}}
\expandafter\def\csname PY@tok@kn\endcsname{\let\PY@bf=\textbf\def\PY@tc##1{\textcolor[rgb]{0.00,0.50,0.00}{##1}}}
\expandafter\def\csname PY@tok@kr\endcsname{\let\PY@bf=\textbf\def\PY@tc##1{\textcolor[rgb]{0.00,0.50,0.00}{##1}}}
\expandafter\def\csname PY@tok@bp\endcsname{\def\PY@tc##1{\textcolor[rgb]{0.00,0.50,0.00}{##1}}}
\expandafter\def\csname PY@tok@fm\endcsname{\def\PY@tc##1{\textcolor[rgb]{0.00,0.00,1.00}{##1}}}
\expandafter\def\csname PY@tok@vc\endcsname{\def\PY@tc##1{\textcolor[rgb]{0.10,0.09,0.49}{##1}}}
\expandafter\def\csname PY@tok@vg\endcsname{\def\PY@tc##1{\textcolor[rgb]{0.10,0.09,0.49}{##1}}}
\expandafter\def\csname PY@tok@vi\endcsname{\def\PY@tc##1{\textcolor[rgb]{0.10,0.09,0.49}{##1}}}
\expandafter\def\csname PY@tok@vm\endcsname{\def\PY@tc##1{\textcolor[rgb]{0.10,0.09,0.49}{##1}}}
\expandafter\def\csname PY@tok@sa\endcsname{\def\PY@tc##1{\textcolor[rgb]{0.73,0.13,0.13}{##1}}}
\expandafter\def\csname PY@tok@sb\endcsname{\def\PY@tc##1{\textcolor[rgb]{0.73,0.13,0.13}{##1}}}
\expandafter\def\csname PY@tok@sc\endcsname{\def\PY@tc##1{\textcolor[rgb]{0.73,0.13,0.13}{##1}}}
\expandafter\def\csname PY@tok@dl\endcsname{\def\PY@tc##1{\textcolor[rgb]{0.73,0.13,0.13}{##1}}}
\expandafter\def\csname PY@tok@s2\endcsname{\def\PY@tc##1{\textcolor[rgb]{0.73,0.13,0.13}{##1}}}
\expandafter\def\csname PY@tok@sh\endcsname{\def\PY@tc##1{\textcolor[rgb]{0.73,0.13,0.13}{##1}}}
\expandafter\def\csname PY@tok@s1\endcsname{\def\PY@tc##1{\textcolor[rgb]{0.73,0.13,0.13}{##1}}}
\expandafter\def\csname PY@tok@mb\endcsname{\def\PY@tc##1{\textcolor[rgb]{0.40,0.40,0.40}{##1}}}
\expandafter\def\csname PY@tok@mf\endcsname{\def\PY@tc##1{\textcolor[rgb]{0.40,0.40,0.40}{##1}}}
\expandafter\def\csname PY@tok@mh\endcsname{\def\PY@tc##1{\textcolor[rgb]{0.40,0.40,0.40}{##1}}}
\expandafter\def\csname PY@tok@mi\endcsname{\def\PY@tc##1{\textcolor[rgb]{0.40,0.40,0.40}{##1}}}
\expandafter\def\csname PY@tok@il\endcsname{\def\PY@tc##1{\textcolor[rgb]{0.40,0.40,0.40}{##1}}}
\expandafter\def\csname PY@tok@mo\endcsname{\def\PY@tc##1{\textcolor[rgb]{0.40,0.40,0.40}{##1}}}
\expandafter\def\csname PY@tok@ch\endcsname{\let\PY@it=\textit\def\PY@tc##1{\textcolor[rgb]{0.25,0.50,0.50}{##1}}}
\expandafter\def\csname PY@tok@cm\endcsname{\let\PY@it=\textit\def\PY@tc##1{\textcolor[rgb]{0.25,0.50,0.50}{##1}}}
\expandafter\def\csname PY@tok@cpf\endcsname{\let\PY@it=\textit\def\PY@tc##1{\textcolor[rgb]{0.25,0.50,0.50}{##1}}}
\expandafter\def\csname PY@tok@c1\endcsname{\let\PY@it=\textit\def\PY@tc##1{\textcolor[rgb]{0.25,0.50,0.50}{##1}}}
\expandafter\def\csname PY@tok@cs\endcsname{\let\PY@it=\textit\def\PY@tc##1{\textcolor[rgb]{0.25,0.50,0.50}{##1}}}

\def\PYZbs{\char`\\}
\def\PYZus{\char`\_}
\def\PYZob{\char`\{}
\def\PYZcb{\char`\}}
\def\PYZca{\char`\^}
\def\PYZam{\char`\&}
\def\PYZlt{\char`\<}
\def\PYZgt{\char`\>}
\def\PYZsh{\char`\#}
\def\PYZpc{\char`\%}
\def\PYZdl{\char`\$}
\def\PYZhy{\char`\-}
\def\PYZsq{\char`\'}
\def\PYZdq{\char`\"}
\def\PYZti{\char`\~}
% for compatibility with earlier versions
\def\PYZat{@}
\def\PYZlb{[}
\def\PYZrb{]}
\makeatother


    % Exact colors from NB
    \definecolor{incolor}{rgb}{0.0, 0.0, 0.5}
    \definecolor{outcolor}{rgb}{0.545, 0.0, 0.0}



    
    % Prevent overflowing lines due to hard-to-break entities
    \sloppy 
    % Setup hyperref package
    \hypersetup{
      breaklinks=true,  % so long urls are correctly broken across lines
      colorlinks=true,
      urlcolor=urlcolor,
      linkcolor=linkcolor,
      citecolor=citecolor,
      }
    % Slightly bigger margins than the latex defaults
    
    \geometry{verbose,tmargin=1in,bmargin=1in,lmargin=1in,rmargin=1in}
    
    

    \begin{document}
    
    
    \maketitle
    
    

    
    \hypertarget{projet-3---sparsituxe9-estimation-et-suxe9lection-de-variables}{%
\section{Projet 3 - Sparsité, estimation et sélection de
variables}\label{projet-3---sparsituxe9-estimation-et-suxe9lection-de-variables}}

    \begin{Verbatim}[commandchars=\\\{\}]
{\color{incolor}In [{\color{incolor}1}]:} \PY{k+kn}{import} \PY{n+nn}{pandas} \PY{k}{as} \PY{n+nn}{pd} 
        \PY{k+kn}{import} \PY{n+nn}{numpy} \PY{k}{as} \PY{n+nn}{np}
        \PY{k+kn}{import} \PY{n+nn}{matplotlib}\PY{n+nn}{.}\PY{n+nn}{pyplot} \PY{k}{as} \PY{n+nn}{plt}
\end{Verbatim}


    \hypertarget{exercice-1}{%
\subsection{Exercice 1}\label{exercice-1}}

    On simule ici n variables aléatoires suivant le modèle de suite
gaussienne soit: \[y_i=a*\eta_j +\xi_j \text{ avec} j=1,...,d \]
\[\text{avec }  \xi_j \sim N[0,1] \]
\[ \text{ et } \eta_i \text{ tel que} \sum_{j=1}^d \eta_j= [d^{1-\beta}] \]
On cherche à déterminer le vecteur sparse theta à partir de différents
estimateurs par seuillage:

1)Estimateur fort: \[\hat{\theta}_j^H = y_i\mathbb{1}(|y_i| > \tau) \]

\begin{enumerate}
\def\labelenumi{\arabic{enumi})}
\setcounter{enumi}{1}
\item
  Estimateur faible:
  \[\hat{\theta}_j^S = y_i\mathbb{1}(1-\frac{\tau}{|y_i|})_+ \]
\item
  Non Negative Garrotte:
  \[\hat{\theta}_j^{NG} = y_i\mathbb{1}(1-\frac{\tau^2}{|y_i|^2})_+ \]
\end{enumerate}

    \begin{Verbatim}[commandchars=\\\{\}]
{\color{incolor}In [{\color{incolor}2}]:} \PY{n}{d} \PY{o}{=} \PY{l+m+mi}{50}
        \PY{n}{beta} \PY{o}{=} \PY{l+m+mf}{0.3}
        \PY{n}{tau} \PY{o}{=} \PY{n}{np}\PY{o}{.}\PY{n}{sqrt}\PY{p}{(}\PY{l+m+mi}{2} \PY{o}{*} \PY{n}{np}\PY{o}{.}\PY{n}{log}\PY{p}{(}\PY{n}{d}\PY{p}{)}\PY{p}{)}
        
        \PY{k}{def} \PY{n+nf}{y}\PY{p}{(}\PY{n}{a}\PY{p}{,} \PY{n}{eta}\PY{p}{,} \PY{n}{ksi}\PY{p}{)}\PY{p}{:}
            \PY{k}{return} \PY{n}{a} \PY{o}{*} \PY{n}{eta} \PY{o}{+} \PY{n}{ksi}
        
        \PY{k}{def} \PY{n+nf}{theta\PYZus{}chap\PYZus{}H}\PY{p}{(}\PY{n}{y}\PY{p}{,} \PY{n}{tau}\PY{p}{)}\PY{p}{:}
            \PY{k}{return} \PY{n}{np}\PY{o}{.}\PY{n}{array}\PY{p}{(}\PY{n}{y} \PY{o}{*} \PY{p}{(}\PY{n}{np}\PY{o}{.}\PY{n}{abs}\PY{p}{(}\PY{n}{y}\PY{p}{)} \PY{o}{\PYZgt{}} \PY{n}{tau} \PY{o}{*} \PY{l+m+mi}{1}\PY{p}{)}\PY{p}{)}
        
        \PY{k}{def} \PY{n+nf}{theta\PYZus{}chap\PYZus{}S}\PY{p}{(}\PY{n}{y}\PY{p}{,} \PY{n}{tau}\PY{p}{)}\PY{p}{:}
            \PY{n}{s} \PY{o}{=} \PY{n}{np}\PY{o}{.}\PY{n}{array}\PY{p}{(}\PY{l+m+mi}{1} \PY{o}{\PYZhy{}} \PY{n}{tau}\PY{o}{/}\PY{n}{np}\PY{o}{.}\PY{n}{abs}\PY{p}{(}\PY{n}{y}\PY{p}{)}\PY{p}{)}
            \PY{c+c1}{\PYZsh{}s[s \PYZlt{} 0] = 0}
            \PY{n}{s} \PY{o}{=} \PY{n}{np}\PY{o}{.}\PY{n}{where}\PY{p}{(}\PY{n}{s} \PY{o}{\PYZlt{}} \PY{l+m+mi}{0}\PY{p}{,} \PY{l+m+mi}{0}\PY{p}{,} \PY{n}{s}\PY{p}{)}
            \PY{n}{s} \PY{o}{=} \PY{n}{y} \PY{o}{*} \PY{n}{s}
            \PY{k}{return} \PY{n}{s}
        
        \PY{k}{def} \PY{n+nf}{theta\PYZus{}chap\PYZus{}NG}\PY{p}{(}\PY{n}{y}\PY{p}{,} \PY{n}{tau}\PY{p}{)}\PY{p}{:}
            \PY{n}{ng} \PY{o}{=} \PY{n}{np}\PY{o}{.}\PY{n}{array}\PY{p}{(}\PY{l+m+mi}{1} \PY{o}{\PYZhy{}} \PY{n}{np}\PY{o}{.}\PY{n}{power}\PY{p}{(}\PY{n}{tau}\PY{p}{,} \PY{l+m+mi}{2}\PY{p}{)}\PY{o}{/}\PY{n}{np}\PY{o}{.}\PY{n}{power}\PY{p}{(}\PY{n}{y}\PY{p}{,} \PY{l+m+mi}{2}\PY{p}{)}\PY{p}{)}
            \PY{c+c1}{\PYZsh{}ng[ng \PYZlt{} 0] = 0}
            \PY{n}{ng} \PY{o}{=} \PY{n}{np}\PY{o}{.}\PY{n}{where}\PY{p}{(}\PY{n}{ng} \PY{o}{\PYZlt{}} \PY{l+m+mi}{0}\PY{p}{,} \PY{l+m+mi}{0}\PY{p}{,} \PY{n}{ng}\PY{p}{)}
            \PY{n}{ng} \PY{o}{=} \PY{n}{y} \PY{o}{*} \PY{n}{ng}
            \PY{k}{return} \PY{n}{ng}
\end{Verbatim}


    1)On trace dans 1er temps, le graphe des 3 estimateurs avec un y variant
entre -10 et 10

    \begin{Verbatim}[commandchars=\\\{\}]
{\color{incolor}In [{\color{incolor}3}]:} \PY{n}{space} \PY{o}{=} \PY{n}{pd}\PY{o}{.}\PY{n}{Series}\PY{p}{(}\PY{n}{np}\PY{o}{.}\PY{n}{linspace}\PY{p}{(}\PY{o}{\PYZhy{}}\PY{l+m+mi}{10}\PY{p}{,} \PY{l+m+mi}{10}\PY{p}{,} \PY{l+m+mi}{1000}\PY{p}{)}\PY{p}{)}
        \PY{n}{theta\PYZus{}chap\PYZus{}H\PYZus{}graph} \PY{o}{=} \PY{n}{theta\PYZus{}chap\PYZus{}H}\PY{p}{(}\PY{n}{space}\PY{p}{,} \PY{n}{tau}\PY{p}{)}
        \PY{n}{theta\PYZus{}chap\PYZus{}S\PYZus{}graph} \PY{o}{=} \PY{n}{theta\PYZus{}chap\PYZus{}S}\PY{p}{(}\PY{n}{space}\PY{p}{,} \PY{n}{tau}\PY{p}{)}
        \PY{n}{theta\PYZus{}chap\PYZus{}NG\PYZus{}graph} \PY{o}{=} \PY{n}{theta\PYZus{}chap\PYZus{}NG}\PY{p}{(}\PY{n}{space}\PY{p}{,} \PY{n}{tau}\PY{p}{)}
        
        \PY{c+c1}{\PYZsh{} Affichage des 3 graphes}
        \PY{n}{plt}\PY{o}{.}\PY{n}{figure}\PY{p}{(}\PY{p}{)}
        \PY{n}{plt}\PY{o}{.}\PY{n}{title}\PY{p}{(}\PY{l+s+s2}{\PYZdq{}}\PY{l+s+s2}{Affichage des 3 estimateurs}\PY{l+s+s2}{\PYZdq{}}\PY{p}{)}
        \PY{n}{plt}\PY{o}{.}\PY{n}{xlabel}\PY{p}{(}\PY{l+s+s2}{\PYZdq{}}\PY{l+s+s2}{y}\PY{l+s+s2}{\PYZdq{}}\PY{p}{)}
        \PY{n}{plt}\PY{o}{.}\PY{n}{ylabel}\PY{p}{(}\PY{l+s+s2}{\PYZdq{}}\PY{l+s+s2}{estimate theta}\PY{l+s+s2}{\PYZdq{}}\PY{p}{)}
        \PY{n}{plt}\PY{o}{.}\PY{n}{plot}\PY{p}{(}\PY{n}{np}\PY{o}{.}\PY{n}{linspace}\PY{p}{(}\PY{o}{\PYZhy{}}\PY{l+m+mi}{10}\PY{p}{,} \PY{l+m+mi}{10}\PY{p}{,} \PY{l+m+mi}{1000}\PY{p}{)}\PY{p}{,} \PY{n}{theta\PYZus{}chap\PYZus{}H\PYZus{}graph}\PY{p}{)}
        \PY{n}{plt}\PY{o}{.}\PY{n}{plot}\PY{p}{(}\PY{n}{np}\PY{o}{.}\PY{n}{linspace}\PY{p}{(}\PY{o}{\PYZhy{}}\PY{l+m+mi}{10}\PY{p}{,} \PY{l+m+mi}{10}\PY{p}{,} \PY{l+m+mi}{1000}\PY{p}{)}\PY{p}{,} \PY{n}{theta\PYZus{}chap\PYZus{}S\PYZus{}graph}\PY{p}{)}
        \PY{n}{plt}\PY{o}{.}\PY{n}{plot}\PY{p}{(}\PY{n}{np}\PY{o}{.}\PY{n}{linspace}\PY{p}{(}\PY{o}{\PYZhy{}}\PY{l+m+mi}{10}\PY{p}{,} \PY{l+m+mi}{10}\PY{p}{,} \PY{l+m+mi}{1000}\PY{p}{)}\PY{p}{,} \PY{n}{theta\PYZus{}chap\PYZus{}NG\PYZus{}graph}\PY{p}{)}
        \PY{n}{plt}\PY{o}{.}\PY{n}{legend}\PY{p}{(}\PY{p}{[}\PY{l+s+s2}{\PYZdq{}}\PY{l+s+s2}{Hard Thresholding}\PY{l+s+s2}{\PYZdq{}}\PY{p}{,}
                   \PY{l+s+s2}{\PYZdq{}}\PY{l+s+s2}{Soft Thresholding}\PY{l+s+s2}{\PYZdq{}}\PY{p}{,}
                   \PY{l+s+s2}{\PYZdq{}}\PY{l+s+s2}{Non\PYZhy{}Negative Garotte}\PY{l+s+s2}{\PYZdq{}}\PY{p}{]}\PY{p}{)}
        \PY{n}{plt}\PY{o}{.}\PY{n}{show}\PY{p}{(}\PY{p}{)}
\end{Verbatim}


    \begin{center}
    \adjustimage{max size={0.9\linewidth}{0.9\paperheight}}{output_6_0.png}
    \end{center}
    { \hspace*{\fill} \\}
    
    On vérifie bien qu'autour de 0, les valeurs retournées sont nulles ce
qui est recherché lors de la construction d'un estimateur sparse.

    2)On détermine ici le risque quadratique des estimateurs. On rappelle:
\[R(\hat{\theta},a)=||\hat{\theta}-\theta^*||_2^2 \]

    \begin{Verbatim}[commandchars=\\\{\}]
{\color{incolor}In [{\color{incolor}4}]:} \PY{k}{def} \PY{n+nf}{simul\PYZus{}y}\PY{p}{(}\PY{n}{d}\PY{p}{,} \PY{n}{beta}\PY{p}{,} \PY{n}{a}\PY{p}{)}\PY{p}{:}
            \PY{c+c1}{\PYZsh{} eta = \PYZob{}0, 1\PYZcb{}}
            \PY{c+c1}{\PYZsh{} Compute the sum of eta to get number of positive eta}
            \PY{n}{sum\PYZus{}eta} \PY{o}{=} \PY{n+nb}{int}\PY{p}{(}\PY{n}{np}\PY{o}{.}\PY{n}{power}\PY{p}{(}\PY{n}{d}\PY{p}{,} \PY{l+m+mi}{1}\PY{o}{\PYZhy{}}\PY{n}{beta}\PY{p}{)}\PY{p}{)}
            \PY{n}{ksi} \PY{o}{=} \PY{p}{[}\PY{p}{]}
            \PY{n}{Y} \PY{o}{=} \PY{p}{[}\PY{p}{]}
            \PY{n}{theta\PYZus{}star} \PY{o}{=} \PY{p}{[}\PY{p}{]}
            \PY{c+c1}{\PYZsh{} Initialize the etas vector (d,1) with only zeros}
            \PY{n}{eta} \PY{o}{=} \PY{n}{np}\PY{o}{.}\PY{n}{repeat}\PY{p}{(}\PY{l+m+mi}{0}\PY{p}{,} \PY{n}{d}\PY{p}{)}
            \PY{c+c1}{\PYZsh{} fill etas vector with ones, in order to have exactly sum\PYZus{}eta \PYZdq{}1\PYZdq{} and (d \PYZhy{} sum\PYZus{}eta) \PYZdq{}0\PYZdq{}}
            \PY{k}{for} \PY{n}{i} \PY{o+ow}{in} \PY{n+nb}{range}\PY{p}{(}\PY{l+m+mi}{1}\PY{p}{,} \PY{n}{sum\PYZus{}eta} \PY{o}{+} \PY{l+m+mi}{1}\PY{p}{)}\PY{p}{:}
                \PY{n}{one\PYZus{}allocated} \PY{o}{=} \PY{k+kc}{False}
                \PY{k}{while} \PY{o+ow}{not} \PY{n}{one\PYZus{}allocated}\PY{p}{:}            
                    \PY{n}{rnd} \PY{o}{=} \PY{n+nb}{int}\PY{p}{(}\PY{n}{np}\PY{o}{.}\PY{n}{random}\PY{o}{.}\PY{n}{random}\PY{p}{(}\PY{p}{)} \PY{o}{*} \PY{n}{d}\PY{p}{)}
                    \PY{n}{one\PYZus{}allocated} \PY{o}{=} \PY{p}{(}\PY{n}{eta}\PY{p}{[}\PY{n}{rnd}\PY{p}{]} \PY{o}{==} \PY{l+m+mi}{0}\PY{p}{)}
                    \PY{n}{eta}\PY{p}{[}\PY{n}{rnd}\PY{p}{]} \PY{o}{=} \PY{l+m+mi}{1}
            \PY{c+c1}{\PYZsh{} Simulating Y}
            \PY{k}{for} \PY{n}{j} \PY{o+ow}{in} \PY{n+nb}{range}\PY{p}{(}\PY{l+m+mi}{0}\PY{p}{,} \PY{n}{d}\PY{p}{)}\PY{p}{:}
                \PY{n}{ksi}\PY{o}{.}\PY{n}{append}\PY{p}{(}\PY{n}{np}\PY{o}{.}\PY{n}{random}\PY{o}{.}\PY{n}{normal}\PY{p}{(}\PY{l+m+mi}{0}\PY{p}{,} \PY{l+m+mi}{1}\PY{p}{,} \PY{l+m+mi}{1}\PY{p}{)}\PY{p}{[}\PY{l+m+mi}{0}\PY{p}{]}\PY{p}{)}
                \PY{n}{Y}\PY{o}{.}\PY{n}{append}\PY{p}{(}\PY{n}{a} \PY{o}{*} \PY{n}{eta}\PY{p}{[}\PY{n}{j}\PY{p}{]} \PY{o}{+} \PY{n}{ksi}\PY{p}{[}\PY{n}{j}\PY{p}{]}\PY{p}{)}
                \PY{n}{theta\PYZus{}star}\PY{o}{.}\PY{n}{append}\PY{p}{(}\PY{n}{a} \PY{o}{*} \PY{n}{eta}\PY{p}{[}\PY{n}{j}\PY{p}{]}\PY{p}{)}
            
            \PY{k}{return} \PY{n}{pd}\PY{o}{.}\PY{n}{Series}\PY{p}{(}\PY{n}{Y}\PY{p}{)}\PY{p}{,} \PY{n}{pd}\PY{o}{.}\PY{n}{Series}\PY{p}{(}\PY{n}{theta\PYZus{}star}\PY{p}{)}
        
        \PY{k}{def} \PY{n+nf}{quad\PYZus{}risk}\PY{p}{(}\PY{n}{Y}\PY{p}{,} \PY{n}{theta\PYZus{}star}\PY{p}{,} \PY{n}{theta\PYZus{}chap\PYZus{}func}\PY{p}{,} \PY{n}{tau}\PY{p}{)}\PY{p}{:}
            \PY{n}{empirical\PYZus{}risk} \PY{o}{=} \PY{l+m+mi}{0}
            \PY{k}{for} \PY{n}{j} \PY{o+ow}{in} \PY{n+nb}{range}\PY{p}{(}\PY{l+m+mi}{0}\PY{p}{,} \PY{n+nb}{len}\PY{p}{(}\PY{n}{Y}\PY{p}{)}\PY{p}{)}\PY{p}{:}
                \PY{n}{empirical\PYZus{}risk} \PY{o}{+}\PY{o}{=} \PY{n}{np}\PY{o}{.}\PY{n}{power}\PY{p}{(}\PY{n}{theta\PYZus{}chap\PYZus{}func}\PY{p}{(}\PY{n}{Y}\PY{p}{[}\PY{n}{j}\PY{p}{]}\PY{p}{,} \PY{n}{tau}\PY{p}{)} \PY{o}{\PYZhy{}} \PY{n}{theta\PYZus{}star}\PY{p}{[}\PY{n}{j}\PY{p}{]}\PY{p}{,} \PY{l+m+mi}{2}\PY{p}{)}
            
            \PY{k}{return} \PY{n}{empirical\PYZus{}risk}
\end{Verbatim}


    \begin{Verbatim}[commandchars=\\\{\}]
{\color{incolor}In [{\color{incolor}5}]:} \PY{n}{estimators} \PY{o}{=} \PY{p}{[}\PY{n}{theta\PYZus{}chap\PYZus{}H}\PY{p}{,} \PY{n}{theta\PYZus{}chap\PYZus{}S}\PY{p}{,} \PY{n}{theta\PYZus{}chap\PYZus{}NG}\PY{p}{]}
        \PY{n}{a\PYZus{}list} \PY{o}{=} \PY{n}{np}\PY{o}{.}\PY{n}{linspace}\PY{p}{(}\PY{l+m+mi}{1}\PY{p}{,} \PY{l+m+mi}{10}\PY{p}{,} \PY{l+m+mi}{100}\PY{p}{)}
        \PY{n}{nb\PYZus{}simul} \PY{o}{=} \PY{l+m+mi}{10}
        
        \PY{n}{all\PYZus{}estimators\PYZus{}graphs} \PY{o}{=} \PY{p}{\PYZob{}}\PY{p}{\PYZcb{}}
        \PY{k}{for} \PY{n}{estimator} \PY{o+ow}{in} \PY{n}{estimators}\PY{p}{:}
            \PY{n}{estimator\PYZus{}graph} \PY{o}{=} \PY{p}{[}\PY{p}{]}
            \PY{k}{for} \PY{n}{a} \PY{o+ow}{in} \PY{n}{a\PYZus{}list}\PY{p}{:}
                \PY{n}{quad\PYZus{}risk\PYZus{}result} \PY{o}{=} \PY{l+m+mi}{0}
                \PY{k}{for} \PY{n}{i\PYZus{}simul} \PY{o+ow}{in} \PY{n+nb}{range}\PY{p}{(}\PY{l+m+mi}{1}\PY{p}{,} \PY{n}{nb\PYZus{}simul}\PY{p}{)}\PY{p}{:}
                    \PY{n}{Y}\PY{p}{,} \PY{n}{theta\PYZus{}star} \PY{o}{=} \PY{n}{simul\PYZus{}y}\PY{p}{(}\PY{n}{d}\PY{p}{,} \PY{n}{beta}\PY{p}{,} \PY{n}{a}\PY{p}{)}
                    \PY{n}{quad\PYZus{}risk\PYZus{}result} \PY{o}{+}\PY{o}{=} \PY{n}{quad\PYZus{}risk}\PY{p}{(}\PY{n}{Y}\PY{p}{,} \PY{n}{theta\PYZus{}star}\PY{p}{,} \PY{n}{estimator}\PY{p}{,} \PY{n}{tau}\PY{p}{)}
                \PY{n}{quad\PYZus{}risk\PYZus{}result} \PY{o}{/}\PY{o}{=} \PY{n}{nb\PYZus{}simul}
                \PY{n}{estimator\PYZus{}graph}\PY{o}{.}\PY{n}{append}\PY{p}{(}\PY{n}{quad\PYZus{}risk\PYZus{}result}\PY{p}{)}
            \PY{n}{all\PYZus{}estimators\PYZus{}graphs}\PY{o}{.}\PY{n}{update}\PY{p}{(}\PY{p}{\PYZob{}}\PY{n}{estimator}\PY{o}{.}\PY{n+nv+vm}{\PYZus{}\PYZus{}name\PYZus{}\PYZus{}}\PY{p}{:}
                                        \PY{n}{estimator\PYZus{}graph}\PY{p}{\PYZcb{}}\PY{p}{)}
        
        \PY{n}{plt}\PY{o}{.}\PY{n}{figure}\PY{p}{(}\PY{p}{)}
        \PY{n}{plt}\PY{o}{.}\PY{n}{title}\PY{p}{(}\PY{l+s+s2}{\PYZdq{}}\PY{l+s+s2}{Risque Quadratique des 3 estimateurs}\PY{l+s+s2}{\PYZdq{}}\PY{p}{)}
        \PY{n}{plt}\PY{o}{.}\PY{n}{xlabel}\PY{p}{(}\PY{l+s+s2}{\PYZdq{}}\PY{l+s+s2}{a}\PY{l+s+s2}{\PYZdq{}}\PY{p}{)}
        \PY{n}{plt}\PY{o}{.}\PY{n}{ylabel}\PY{p}{(}\PY{l+s+s2}{\PYZdq{}}\PY{l+s+s2}{R(theta, a)}\PY{l+s+s2}{\PYZdq{}}\PY{p}{)}
        \PY{n}{legends} \PY{o}{=} \PY{p}{[}\PY{p}{]}
        \PY{k}{for} \PY{n}{estimator\PYZus{}name}\PY{p}{,} \PY{n}{estimator\PYZus{}graph} \PY{o+ow}{in} \PY{n}{all\PYZus{}estimators\PYZus{}graphs}\PY{o}{.}\PY{n}{items}\PY{p}{(}\PY{p}{)}\PY{p}{:}
            \PY{n}{plt}\PY{o}{.}\PY{n}{plot}\PY{p}{(}\PY{n}{a\PYZus{}list}\PY{p}{,} \PY{n}{estimator\PYZus{}graph}\PY{p}{)}
            \PY{n}{legends}\PY{o}{.}\PY{n}{append}\PY{p}{(}\PY{l+s+s2}{\PYZdq{}}\PY{l+s+s2}{theta = }\PY{l+s+s2}{\PYZdq{}} \PY{o}{+} \PY{n}{estimator\PYZus{}name}\PY{p}{)}
        \PY{n}{plt}\PY{o}{.}\PY{n}{legend}\PY{p}{(}\PY{n}{legends}\PY{p}{)}
        \PY{n}{plt}\PY{o}{.}\PY{n}{show}\PY{p}{(}\PY{p}{)}
\end{Verbatim}


    \begin{center}
    \adjustimage{max size={0.9\linewidth}{0.9\paperheight}}{output_10_0.png}
    \end{center}
    { \hspace*{\fill} \\}
    
    On observe que le risque quadratique tend vers 0 pour l'estimateur
faible et non-negative garrotte mais qu'il tend vers 130 pour
l'estimateur fort.

    3)On détermine ici le risque de sélection de variables pour chaqu'un des
estimateurs. Ce risque s'exprime comme:
\[ R^{MS}(\hat{\theta},a)=\sum_{j=1}^M |\eta_j -\hat{\eta_j}| \]
\[\text{avec } \hat{\eta_j}=\mathbb{1}(|y_i| \geqslant \tau) \]

    \begin{Verbatim}[commandchars=\\\{\}]
{\color{incolor}In [{\color{incolor}6}]:} \PY{k}{def} \PY{n+nf}{selection\PYZus{}var\PYZus{}risk}\PY{p}{(}\PY{n}{eta}\PY{p}{,} \PY{n}{eta\PYZus{}chap}\PY{p}{)}\PY{p}{:}
            \PY{n}{sum\PYZus{}risk} \PY{o}{=} \PY{l+m+mi}{0}
            \PY{k}{for} \PY{n}{j} \PY{o+ow}{in} \PY{n+nb}{range}\PY{p}{(}\PY{l+m+mi}{0}\PY{p}{,} \PY{n+nb}{len}\PY{p}{(}\PY{n}{eta}\PY{p}{)}\PY{p}{)}\PY{p}{:}
                \PY{n}{sum\PYZus{}risk} \PY{o}{+}\PY{o}{=} \PY{n}{np}\PY{o}{.}\PY{n}{abs}\PY{p}{(}\PY{n}{eta}\PY{p}{[}\PY{n}{j}\PY{p}{]} \PY{o}{\PYZhy{}} \PY{n}{eta\PYZus{}chap}\PY{p}{[}\PY{n}{j}\PY{p}{]}\PY{p}{)}
        
            \PY{k}{return} \PY{n}{sum\PYZus{}risk}
\end{Verbatim}


    \begin{Verbatim}[commandchars=\\\{\}]
{\color{incolor}In [{\color{incolor}7}]:} \PY{n}{estimators} \PY{o}{=} \PY{p}{[}\PY{n}{theta\PYZus{}chap\PYZus{}H}\PY{p}{,} \PY{n}{theta\PYZus{}chap\PYZus{}S}\PY{p}{,} \PY{n}{theta\PYZus{}chap\PYZus{}NG}\PY{p}{]}
        \PY{n}{a\PYZus{}list} \PY{o}{=} \PY{n}{np}\PY{o}{.}\PY{n}{linspace}\PY{p}{(}\PY{l+m+mi}{1}\PY{p}{,} \PY{l+m+mi}{10}\PY{p}{,} \PY{l+m+mi}{100}\PY{p}{)}
        \PY{n}{nb\PYZus{}simul} \PY{o}{=} \PY{l+m+mi}{10}
        
        \PY{n}{all\PYZus{}risk\PYZus{}graph} \PY{o}{=} \PY{p}{\PYZob{}}\PY{p}{\PYZcb{}}
        \PY{k}{for} \PY{n}{estimator} \PY{o+ow}{in} \PY{n}{estimators}\PY{p}{:}
            \PY{n}{risk\PYZus{}graph} \PY{o}{=} \PY{p}{[}\PY{p}{]}
            \PY{k}{for} \PY{n}{a} \PY{o+ow}{in} \PY{n}{a\PYZus{}list}\PY{p}{:}
                \PY{n}{selection\PYZus{}var\PYZus{}risk\PYZus{}result} \PY{o}{=} \PY{l+m+mi}{0}
                \PY{k}{for} \PY{n}{i\PYZus{}simul} \PY{o+ow}{in} \PY{n+nb}{range}\PY{p}{(}\PY{l+m+mi}{1}\PY{p}{,} \PY{n}{nb\PYZus{}simul}\PY{p}{)}\PY{p}{:}
                    \PY{n}{Y}\PY{p}{,} \PY{n}{theta\PYZus{}star} \PY{o}{=} \PY{n}{simul\PYZus{}y}\PY{p}{(}\PY{n}{d}\PY{p}{,} \PY{n}{beta}\PY{p}{,} \PY{n}{a}\PY{p}{)}
                    \PY{n}{eta} \PY{o}{=} \PY{n}{theta\PYZus{}star}\PY{o}{/}\PY{n}{a}
                    \PY{n}{eta\PYZus{}chap} \PY{o}{=} \PY{p}{(}\PY{n}{np}\PY{o}{.}\PY{n}{abs}\PY{p}{(}\PY{n}{Y}\PY{p}{)} \PY{o}{\PYZgt{}}\PY{o}{=} \PY{n}{tau}\PY{p}{)} \PY{o}{*} \PY{l+m+mi}{1}
                    \PY{n}{selection\PYZus{}var\PYZus{}risk\PYZus{}result} \PY{o}{+}\PY{o}{=} \PY{n}{selection\PYZus{}var\PYZus{}risk}\PY{p}{(}\PY{n}{eta}\PY{p}{,} \PY{n}{eta\PYZus{}chap}\PY{p}{)}
                \PY{n}{selection\PYZus{}var\PYZus{}risk\PYZus{}result} \PY{o}{/}\PY{o}{=} \PY{n}{nb\PYZus{}simul}
                \PY{n}{risk\PYZus{}graph}\PY{o}{.}\PY{n}{append}\PY{p}{(}\PY{n}{selection\PYZus{}var\PYZus{}risk\PYZus{}result}\PY{p}{)}
            \PY{n}{all\PYZus{}risk\PYZus{}graph}\PY{o}{.}\PY{n}{update}\PY{p}{(}\PY{p}{\PYZob{}}\PY{n}{estimator}\PY{o}{.}\PY{n+nv+vm}{\PYZus{}\PYZus{}name\PYZus{}\PYZus{}}\PY{p}{:}
                                        \PY{n}{risk\PYZus{}graph}\PY{p}{\PYZcb{}}\PY{p}{)}
        
        \PY{n}{plt}\PY{o}{.}\PY{n}{figure}\PY{p}{(}\PY{p}{)}
        \PY{n}{plt}\PY{o}{.}\PY{n}{title}\PY{p}{(}\PY{l+s+s2}{\PYZdq{}}\PY{l+s+s2}{Risque De Sélection de Variables des 3 estimateurs}\PY{l+s+s2}{\PYZdq{}}\PY{p}{)}
        \PY{n}{plt}\PY{o}{.}\PY{n}{xlabel}\PY{p}{(}\PY{l+s+s2}{\PYZdq{}}\PY{l+s+s2}{a}\PY{l+s+s2}{\PYZdq{}}\PY{p}{)}
        \PY{n}{plt}\PY{o}{.}\PY{n}{ylabel}\PY{p}{(}\PY{l+s+s2}{\PYZdq{}}\PY{l+s+s2}{R\PYZca{}MS(theta, a)}\PY{l+s+s2}{\PYZdq{}}\PY{p}{)}
        \PY{n}{legends} \PY{o}{=} \PY{p}{[}\PY{p}{]}
        \PY{k}{for} \PY{n}{estimator\PYZus{}name}\PY{p}{,} \PY{n}{risk\PYZus{}graph} \PY{o+ow}{in} \PY{n}{all\PYZus{}risk\PYZus{}graph}\PY{o}{.}\PY{n}{items}\PY{p}{(}\PY{p}{)}\PY{p}{:}
            \PY{n}{plt}\PY{o}{.}\PY{n}{plot}\PY{p}{(}\PY{n}{a\PYZus{}list}\PY{p}{,} \PY{n}{risk\PYZus{}graph}\PY{p}{)}
            \PY{n}{legends}\PY{o}{.}\PY{n}{append}\PY{p}{(}\PY{l+s+s2}{\PYZdq{}}\PY{l+s+s2}{theta = }\PY{l+s+s2}{\PYZdq{}} \PY{o}{+} \PY{n}{estimator\PYZus{}name}\PY{p}{)}
        \PY{n}{plt}\PY{o}{.}\PY{n}{legend}\PY{p}{(}\PY{n}{legends}\PY{p}{)}
        \PY{n}{plt}\PY{o}{.}\PY{n}{show}\PY{p}{(}\PY{p}{)}
\end{Verbatim}


    \begin{center}
    \adjustimage{max size={0.9\linewidth}{0.9\paperheight}}{output_14_0.png}
    \end{center}
    { \hspace*{\fill} \\}
    
    On constate que le risque de sélection décroit quand a augmente et que
ce risque est nul lorsque a est supérieur à 5

    \hypertarget{exercice-2}{%
\subsection{Exercice 2}\label{exercice-2}}

    Dans cette exercice on s'interesse à la détection de rupture. On
construit comme avant à des variables aléatoires suivant le modèle de
suite gaussienne avec des thetas construits de façon à créer une
rupture.

    \begin{Verbatim}[commandchars=\\\{\}]
{\color{incolor}In [{\color{incolor}8}]:} \PY{n}{d} \PY{o}{=} \PY{l+m+mi}{50}
        \PY{n}{tau} \PY{o}{=} \PY{n}{np}\PY{o}{.}\PY{n}{sqrt}\PY{p}{(}\PY{l+m+mi}{2}\PY{o}{*}\PY{n}{np}\PY{o}{.}\PY{n}{log}\PY{p}{(}\PY{n}{d}\PY{p}{)}\PY{p}{)}
        
        \PY{n}{theta\PYZus{}star} \PY{o}{=} \PY{n}{np}\PY{o}{.}\PY{n}{repeat}\PY{p}{(}\PY{l+m+mi}{1}\PY{p}{,} \PY{n}{d}\PY{p}{)}
        \PY{n}{theta\PYZus{}star}\PY{p}{[}\PY{l+m+mi}{0}\PY{p}{:}\PY{l+m+mi}{10}\PY{p}{]} \PY{o}{=} \PY{n}{np}\PY{o}{.}\PY{n}{repeat}\PY{p}{(}\PY{l+m+mi}{3}\PY{p}{,} \PY{l+m+mi}{10}\PY{p}{)}
        \PY{n}{theta\PYZus{}star}\PY{p}{[}\PY{l+m+mi}{10}\PY{p}{:}\PY{l+m+mi}{30}\PY{p}{]} \PY{o}{=} \PY{n}{np}\PY{o}{.}\PY{n}{repeat}\PY{p}{(}\PY{l+m+mi}{7}\PY{p}{,} \PY{l+m+mi}{20}\PY{p}{)}
        \PY{n}{theta\PYZus{}star}\PY{p}{[}\PY{l+m+mi}{30}\PY{p}{:}\PY{l+m+mi}{40}\PY{p}{]} \PY{o}{=} \PY{n}{np}\PY{o}{.}\PY{n}{repeat}\PY{p}{(}\PY{l+m+mf}{1.5}\PY{p}{,} \PY{l+m+mi}{10}\PY{p}{)}
        \PY{n}{theta\PYZus{}star}\PY{p}{[}\PY{l+m+mi}{40}\PY{p}{:}\PY{p}{]} \PY{o}{=} \PY{n}{np}\PY{o}{.}\PY{n}{repeat}\PY{p}{(}\PY{l+m+mi}{2}\PY{p}{,} \PY{p}{(}\PY{n}{d}\PY{o}{\PYZhy{}}\PY{l+m+mi}{40}\PY{p}{)}\PY{p}{)}
        
        \PY{n}{theta\PYZus{}star} \PY{o}{=} \PY{n}{pd}\PY{o}{.}\PY{n}{Series}\PY{p}{(}\PY{n}{theta\PYZus{}star}\PY{p}{)}
\end{Verbatim}


    Création du vecteur y

    \begin{Verbatim}[commandchars=\\\{\}]
{\color{incolor}In [{\color{incolor}9}]:} \PY{n}{epsilon} \PY{o}{=} \PY{l+m+mi}{1}\PY{o}{/}\PY{n}{np}\PY{o}{.}\PY{n}{sqrt}\PY{p}{(}\PY{n}{d}\PY{p}{)}
        \PY{n}{y} \PY{o}{=} \PY{n}{theta\PYZus{}star} \PY{o}{+} \PY{n}{epsilon} \PY{o}{*} \PY{n}{np}\PY{o}{.}\PY{n}{random}\PY{o}{.}\PY{n}{normal}\PY{p}{(}\PY{l+m+mi}{0}\PY{p}{,} \PY{l+m+mi}{1}\PY{p}{,} \PY{n}{d}\PY{p}{)}
\end{Verbatim}


    Création de la matrice Delta

    Celle ci est construite comme suit: \[\Delta_j^*=z_{j+1}-z_j \]

    \begin{itemize}
\tightlist
\item
  à partir du vecteur theta --\textgreater{} pas d'aléatoire
\end{itemize}

    \begin{Verbatim}[commandchars=\\\{\}]
{\color{incolor}In [{\color{incolor}10}]:} \PY{n}{delta\PYZus{}star} \PY{o}{=} \PY{n}{theta\PYZus{}star} \PY{o}{\PYZhy{}} \PY{n}{theta\PYZus{}star}\PY{o}{.}\PY{n}{shift}\PY{p}{(}\PY{l+m+mi}{1}\PY{p}{)}
         \PY{n}{delta\PYZus{}star} \PY{o}{=} \PY{n}{delta\PYZus{}star}\PY{o}{.}\PY{n}{dropna}\PY{p}{(}\PY{p}{)}\PY{o}{.}\PY{n}{reset\PYZus{}index}\PY{p}{(}\PY{n}{drop}\PY{o}{=}\PY{k+kc}{True}\PY{p}{)}
\end{Verbatim}


    \begin{itemize}
\tightlist
\item
  à partir du vecteur y --\textgreater{} theta + de l'aléatoire
\end{itemize}

    \begin{Verbatim}[commandchars=\\\{\}]
{\color{incolor}In [{\color{incolor}11}]:} \PY{n}{y\PYZus{}delta} \PY{o}{=} \PY{n}{y} \PY{o}{\PYZhy{}} \PY{n}{y}\PY{o}{.}\PY{n}{shift}\PY{p}{(}\PY{l+m+mi}{1}\PY{p}{)}
         \PY{n}{y\PYZus{}delta} \PY{o}{=} \PY{n}{y\PYZus{}delta}\PY{o}{.}\PY{n}{dropna}\PY{p}{(}\PY{p}{)}\PY{o}{.}\PY{n}{reset\PYZus{}index}\PY{p}{(}\PY{n}{drop}\PY{o}{=}\PY{k+kc}{True}\PY{p}{)}
\end{Verbatim}


    \[\textbf{Estimation par Seuillage Dur des theta chapeau} \]

    Pour estimer J* vecteur sparse des thetas, on utilse les différents
estimateurs de seuillage de l'exercie 1.

    \begin{itemize}
\tightlist
\item
  Estimation de J*
\end{itemize}

    \begin{Verbatim}[commandchars=\\\{\}]
{\color{incolor}In [{\color{incolor}12}]:} \PY{n}{delta\PYZus{}chap} \PY{o}{=} \PY{n}{theta\PYZus{}chap\PYZus{}H}\PY{p}{(}\PY{n}{y\PYZus{}delta}\PY{p}{,} \PY{n}{tau}\PY{p}{)}
         \PY{n}{J\PYZus{}star} \PY{o}{=} \PY{n}{np}\PY{o}{.}\PY{n}{where}\PY{p}{(}\PY{n}{delta\PYZus{}chap} \PY{o}{!=} \PY{l+m+mi}{0}\PY{p}{)}
         \PY{n}{J\PYZus{}star}
\end{Verbatim}


\begin{Verbatim}[commandchars=\\\{\}]
{\color{outcolor}Out[{\color{outcolor}12}]:} (array([ 9, 29], dtype=int64),)
\end{Verbatim}
            
    \begin{itemize}
\tightlist
\item
  Risque Quadratique Estimé
\end{itemize}

    \begin{Verbatim}[commandchars=\\\{\}]
{\color{incolor}In [{\color{incolor}13}]:} \PY{c+c1}{\PYZsh{} méthode 1}
         \PY{n}{est\PYZus{}quad\PYZus{}risk\PYZus{}H} \PY{o}{=} \PY{n}{quad\PYZus{}risk}\PY{p}{(}\PY{n}{y\PYZus{}delta}\PY{p}{,} \PY{n}{delta\PYZus{}star}\PY{p}{,} \PY{n}{theta\PYZus{}chap\PYZus{}H}\PY{p}{,} \PY{n}{tau}\PY{p}{)}
         \PY{n+nb}{print}\PY{p}{(}\PY{n}{est\PYZus{}quad\PYZus{}risk\PYZus{}H}\PY{p}{)}
\end{Verbatim}


    \begin{Verbatim}[commandchars=\\\{\}]
1.1505912079349065

    \end{Verbatim}

    \[\textbf{Estimation par Seuillage Doux des theta chapeau} \]

    \begin{itemize}
\tightlist
\item
  Estimation de J*
\end{itemize}

    \begin{Verbatim}[commandchars=\\\{\}]
{\color{incolor}In [{\color{incolor}14}]:} \PY{n}{delta\PYZus{}chap} \PY{o}{=} \PY{n}{theta\PYZus{}chap\PYZus{}S}\PY{p}{(}\PY{n}{y\PYZus{}delta}\PY{p}{,} \PY{n}{tau}\PY{p}{)}
         \PY{n}{J\PYZus{}star} \PY{o}{=} \PY{n}{np}\PY{o}{.}\PY{n}{where}\PY{p}{(}\PY{n}{delta\PYZus{}chap} \PY{o}{!=} \PY{l+m+mi}{0}\PY{p}{)}
         \PY{n}{J\PYZus{}star}
\end{Verbatim}


\begin{Verbatim}[commandchars=\\\{\}]
{\color{outcolor}Out[{\color{outcolor}14}]:} (array([ 9, 29], dtype=int64),)
\end{Verbatim}
            
    \begin{itemize}
\tightlist
\item
  Risque Quadratique Estimé
\end{itemize}

    \begin{Verbatim}[commandchars=\\\{\}]
{\color{incolor}In [{\color{incolor}15}]:} \PY{c+c1}{\PYZsh{} méthode 1}
         \PY{n}{est\PYZus{}quad\PYZus{}risk\PYZus{}S} \PY{o}{=} \PY{n}{quad\PYZus{}risk}\PY{p}{(}\PY{n}{y\PYZus{}delta}\PY{p}{,} \PY{n}{delta\PYZus{}star}\PY{p}{,}\PY{n}{theta\PYZus{}chap\PYZus{}S}\PY{p}{,} \PY{n}{tau}\PY{p}{)}
         \PY{n+nb}{print}\PY{p}{(}\PY{n}{est\PYZus{}quad\PYZus{}risk\PYZus{}S}\PY{p}{)}
\end{Verbatim}


    \begin{Verbatim}[commandchars=\\\{\}]
14.199422494375856

    \end{Verbatim}

    \[ \textbf{Estimation par Non-Negative Garrotte des theta chapeau}\]

    \begin{itemize}
\tightlist
\item
  Estimation de J*
\end{itemize}

    \begin{Verbatim}[commandchars=\\\{\}]
{\color{incolor}In [{\color{incolor}16}]:} \PY{n}{delta\PYZus{}chap} \PY{o}{=} \PY{n}{theta\PYZus{}chap\PYZus{}NG}\PY{p}{(}\PY{n}{y\PYZus{}delta}\PY{p}{,} \PY{n}{tau}\PY{p}{)}
         \PY{n}{J\PYZus{}star} \PY{o}{=} \PY{n}{np}\PY{o}{.}\PY{n}{where}\PY{p}{(}\PY{n}{delta\PYZus{}chap} \PY{o}{!=} \PY{l+m+mi}{0}\PY{p}{)}
         \PY{n}{J\PYZus{}star}
\end{Verbatim}


\begin{Verbatim}[commandchars=\\\{\}]
{\color{outcolor}Out[{\color{outcolor}16}]:} (array([ 9, 29], dtype=int64),)
\end{Verbatim}
            
    \begin{itemize}
\tightlist
\item
  Risque Quadratique Estimé
\end{itemize}

    \begin{Verbatim}[commandchars=\\\{\}]
{\color{incolor}In [{\color{incolor}17}]:} \PY{c+c1}{\PYZsh{} méthode 1}
         \PY{n}{est\PYZus{}quad\PYZus{}risk\PYZus{}NG} \PY{o}{=} \PY{n}{quad\PYZus{}risk}\PY{p}{(}\PY{n}{y\PYZus{}delta}\PY{p}{,} \PY{n}{delta\PYZus{}star}\PY{p}{,}\PY{n}{theta\PYZus{}chap\PYZus{}NG}\PY{p}{,} \PY{n}{tau}\PY{p}{)}
         \PY{n+nb}{print}\PY{p}{(}\PY{n}{est\PYZus{}quad\PYZus{}risk\PYZus{}NG}\PY{p}{)}
\end{Verbatim}


    \begin{Verbatim}[commandchars=\\\{\}]
5.062795922660021

    \end{Verbatim}

    \[\textbf{Résultat:} \] On constate bien que le vecteur retourné par les
3 estimateurs est (9, 29), ce qui correspond bien à l'indice des deux
points de rupture de nos valeurs.

    \hypertarget{exercice-3}{%
\subsection{Exercice 3}\label{exercice-3}}

    On s'interrese ici au cours de Bourse de Renault pendant le mois
dernier. On cherche à savoir si les révélations sur Carlos Ghosn ont eut
un impact qui a fait chuter d'un coup le cours en bourse.

    \begin{Verbatim}[commandchars=\\\{\}]
{\color{incolor}In [{\color{incolor}123}]:} \PY{n}{df} \PY{o}{=} \PY{n}{pd}\PY{o}{.}\PY{n}{read\PYZus{}excel}\PY{p}{(}\PY{l+s+sa}{r}\PY{l+s+s2}{\PYZdq{}}\PY{l+s+s2}{CoursRenault.xls}\PY{l+s+s2}{\PYZdq{}}\PY{p}{)}
          
          \PY{n}{close} \PY{o}{=} \PY{n}{df}\PY{p}{[}\PY{l+s+s2}{\PYZdq{}}\PY{l+s+s2}{clot}\PY{l+s+s2}{\PYZdq{}}\PY{p}{]}
          \PY{n}{date}\PY{o}{=}\PY{n}{df}\PY{p}{[}\PY{l+s+s2}{\PYZdq{}}\PY{l+s+s2}{date}\PY{l+s+s2}{\PYZdq{}}\PY{p}{]}
          
          
          \PY{n}{d} \PY{o}{=} \PY{n+nb}{len}\PY{p}{(}\PY{n}{close}\PY{p}{)}
          \PY{n}{tau} \PY{o}{=} \PY{n}{np}\PY{o}{.}\PY{n}{sqrt}\PY{p}{(}\PY{l+m+mi}{5}\PY{o}{*}\PY{n}{np}\PY{o}{.}\PY{n}{log}\PY{p}{(}\PY{n}{d}\PY{p}{)}\PY{p}{)}
          \PY{n}{close}\PY{o}{.}\PY{n}{head}\PY{p}{(}\PY{p}{)}
\end{Verbatim}


\begin{Verbatim}[commandchars=\\\{\}]
{\color{outcolor}Out[{\color{outcolor}123}]:} 0    85.23
          1    84.88
          2    83.92
          3    83.51
          4    84.41
          Name: clot, dtype: float64
\end{Verbatim}
            
    \begin{Verbatim}[commandchars=\\\{\}]
{\color{incolor}In [{\color{incolor}124}]:} \PY{n}{close\PYZus{}delta} \PY{o}{=} \PY{n}{close} \PY{o}{\PYZhy{}} \PY{n}{close}\PY{o}{.}\PY{n}{shift}\PY{p}{(}\PY{l+m+mi}{1}\PY{p}{)}
          \PY{n}{close\PYZus{}delta} \PY{o}{=} \PY{n}{close\PYZus{}delta}\PY{o}{.}\PY{n}{dropna}\PY{p}{(}\PY{p}{)}\PY{o}{.}\PY{n}{reset\PYZus{}index}\PY{p}{(}\PY{n}{drop}\PY{o}{=}\PY{k+kc}{True}\PY{p}{)}
\end{Verbatim}


    \begin{Verbatim}[commandchars=\\\{\}]
{\color{incolor}In [{\color{incolor}125}]:} \PY{n}{theta\PYZus{}chap} \PY{o}{=} \PY{n}{theta\PYZus{}chap\PYZus{}H}\PY{p}{(}\PY{n}{close\PYZus{}delta}\PY{p}{,} \PY{n}{tau}\PY{p}{)}
          \PY{n}{J\PYZus{}star} \PY{o}{=} \PY{n}{np}\PY{o}{.}\PY{n}{where}\PY{p}{(}\PY{n}{theta\PYZus{}chap} \PY{o}{!=} \PY{l+m+mi}{0}\PY{p}{)}
          \PY{n+nb}{print}\PY{p}{(}\PY{n}{J\PYZus{}star}\PY{p}{)}
          \PY{n+nb}{print}\PY{p}{(}\PY{n}{date}\PY{p}{[}\PY{n}{J\PYZus{}star}\PY{p}{[}\PY{l+m+mi}{0}\PY{p}{]}\PY{o}{+}\PY{l+m+mi}{1}\PY{p}{]}\PY{p}{)}
\end{Verbatim}


    \begin{Verbatim}[commandchars=\\\{\}]
(array([ 79, 243], dtype=int64),)
80    2018-03-29
244   2018-11-19
Name: date, dtype: datetime64[ns]

    \end{Verbatim}

    On observe bien une rupture du cours de bourse au niveau de la 12ème
valeur du Dataset (à la date du 19/11/2018) ce qui correspond au jour de
sortie des révélations. L'autre point de ruputre du 29 mars correspond à
une rumeur de fusion avec Nissan


    % Add a bibliography block to the postdoc
    
    
    
    \end{document}
